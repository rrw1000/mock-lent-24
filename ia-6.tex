\documentclass{tripos}  % Do not add anything outside the question environment
\begin{document}
\begin{question}[MockIA,year=2024,paper=1,question=6,author=rrw]{Object-Orientated Programming}

You are writing an image processing library in Java. The design is
for an abstract image class which will be specialised for various
tasks.

Your library will need to support images in:

\begin{itemize}
\item RGB or HSV colour
\item Raster or JPEG data representation (you will be provided with a JPEG class that can access individual pixels in directly encoded JPEG binary data)
\item Planar or in-line colour.
\end{itemize}

And will need to support:

\begin{itemize}
\item Filtering each pixels in an image in various orders (row-first, column-first .. )
\item Bit-blitting: take a rectangular part a source image, and a rectangular part of a destination image and modify the destination image pixel-wise with an operation applied to both the source and destination image - eg. XORing a rectangular part of image A with a rectangular part of B, with the result going to B.
\item Loading and saving the images to a file. You are provided with libraries for doing this.
\end{itemize}

\begin{enumerate}
\item Design and describe your library, using appropriate design
  patterns and generics. Give brief pseudocode for how a user of your
  library would invoke the operations above, and whwat design patterns
  you have used.  \fullmarks{10}
\item Draw a UML diagram to represent your library.
  \fullmarks{5}
\item What optimisations can you apply to ensure that your library operates reasonably efficiently? How do your abstractions support this?
  \fullmarks{5}
\end{enumerate}

\end{question}
\end{document}

