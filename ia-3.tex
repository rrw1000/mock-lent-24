\documentclass{tripos}  % Do not add anything outside the question environment
\begin{document}
\begin{question}[MockIA,year=2024,paper=1,question=3,author=rrw]{Discrete Mathematics}

  \begin{enumerate}
   
  \item Without using the Fundamental Theorem of Arithmetic, prove that

    \[
    \mbox{gcd}(c,ab) = 1 \Leftrightarrow ( \mbox{gcd}(c,a) = 1 \wedge{} \mbox{gcd}(c,b) = 1)
    \]

    for all positive integers, $a,b,c$.\fullmarks{4}

  \item Let $I = { x \in R\, | 0 \le x \le 1 }$. Define a function from $I$ to $I$ that is injective but not bijective. \fullmarks{3}
  \item How many relations are there between a (finite) set $A$ and itself? \fullmarks{2}
  \item How many of these are both reflexive and antisymmetric? \fullmarks{3}


  \item Recall that $B_{ij}(X,Y)$ denotes the set of bijections between sets X and Y, and that for $n \in \mathbb{N}$ the set $[n]$ is defined as $\{i \in \mathbb{N}\, |\, i < n\}$
    \begin{enumerate}
    \item Given a set $A$ such that $0 \notin A$, and $Q=\{ 0 \cup A \}$, describe a bijection
      \[
      B_{ij}(Q,Q) \rightarrow ( Q \times B_{ij}(A,A) )
      \]\fullmarks{4}
    \item Hence or otherwise, prove that $\forall n \in \mathbb{N}.| B_{ij}([n], [n]) | \approxeq n!$ \fullmarks{4}
      \end{enumerate}
  \end{enumerate}
  
\end{question}
\end{document}

